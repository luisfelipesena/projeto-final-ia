\documentclass[aspectratio=169,xcolor=table]{beamer}
\usepackage[utf8]{inputenc}
\usepackage[T1]{fontenc}
\usepackage{lmodern}
\usepackage[portuguese]{babel}
\usepackage{hyperref}
\usepackage{tabularx}
\usepackage{amsmath}
\usepackage{amssymb}
\usepackage{tikz}
\usetikzlibrary{shapes,arrows,positioning,fit}

% ------------------------------------------------
% Tema e Configurações do Beamer
% ------------------------------------------------
\usetheme{DCC}

% Ajuste de espaçamento entre itens
\setbeamertemplate{itemize items}[circle]
\setbeamertemplate{itemize subitem}[circle]
\setlength{\itemsep}{0.8em}
\setlength{\parskip}{0.5em}

\graphicspath{{imgs/}{./imgs/}}

\author[Luis Felipe]{%
  \textbf{Luis Felipe Cordeiro Sena}
}
\title{YouBot Autônomo: Coleta e Organização de Cubos}
\subtitle{MATA64 - Inteligência Artificial | Projeto Final}
\institute{Universidade Federal da Bahia \\ Instituto de Computação}
\date{Janeiro 2026}

\begin{document}

%-------------------------------------------------
%  SLIDE DE TÍTULO
%-------------------------------------------------
\begin{frame}[plain,noframenumbering]
    \titlepage
\end{frame}

%-------------------------------------------------
%  SLIDE DE AGENDA
%-------------------------------------------------
\begin{frame}{Agenda}
    \begin{table}
        \begin{tabularx}{\textwidth}{|l|X|}
            \hline
            \textbf{Tempo} & \textbf{Conteúdo} \\
            \hline
            0–3 min  & \textbf{Objetivo e Desafio:} Coleta autônoma de 15 cubos coloridos \\
            \hline
            3–6 min & \textbf{Arquitetura do Sistema:} Pipeline percepção-decisão-ação \\
            \hline
            6–9 min & \textbf{Redes Neurais:} LIDAR CNN + Classificador de Cores \\
            \hline
            9–12 min & \textbf{Lógica Fuzzy:} Controlador com 25 regras \\
            \hline
            12–15 min & \textbf{Demonstração:} Robô em operação autônoma \\
            \hline
        \end{tabularx}
    \end{table}
\end{frame}

%=================================================
\section{Objetivo e Desafio}
%=================================================
\begin{frame}{O Problema}
    \begin{columns}
        \begin{column}{0.5\textwidth}
            \textbf{Tarefa:}
            \begin{itemize}
                \item Coletar \textbf{15 cubos coloridos}
                \item Cores: verde, azul, vermelho
                \item Depositar na caixa correspondente
                \item Evitar obstáculos (caixotes de madeira)
            \end{itemize}

            \vspace{1em}
            \textbf{Restrições:}
            \begin{itemize}
                \item \textbf{SEM GPS} na demonstração final
                \item Spawn aleatório dos cubos
                \item Arena com obstáculos fixos
            \end{itemize}
        \end{column}
        \begin{column}{0.5\textwidth}
            % TODO: Inserir screenshot da arena
            \begin{center}
                \textit{[Figura: Arena de simulação]}

                \vspace{1em}
                YouBot com LIDAR + Câmera RGB
            \end{center}
        \end{column}
    \end{columns}
\end{frame}

\begin{frame}{Sensores Disponíveis}
    \begin{columns}
        \begin{column}{0.5\textwidth}
            \textbf{LIDAR (Hokuyo)}
            \begin{itemize}
                \item 667 pontos por varredura
                \item FOV: 270°
                \item Range: 0.1m - 5.0m
                \item Função: Detecção de obstáculos
            \end{itemize}
        \end{column}
        \begin{column}{0.5\textwidth}
            \textbf{Câmera RGB}
            \begin{itemize}
                \item Resolução: 512×512 pixels
                \item FOV: ~60°
                \item Função: Identificação de cubos e cores
            \end{itemize}
        \end{column}
    \end{columns}

    \vspace{1em}
    \begin{center}
        % TODO: Inserir diagrama de sensores
        \textit{[Figura: Diagrama de posicionamento dos sensores no YouBot]}
    \end{center}
\end{frame}

%=================================================
\section{Arquitetura do Sistema}
%=================================================
\begin{frame}{Pipeline Percepção-Decisão-Ação}
    \begin{center}
    \begin{tikzpicture}[
        node distance=1.5cm,
        block/.style={rectangle, draw, fill=blue!20, text width=2.5cm, text centered, minimum height=1cm},
        arrow/.style={->, >=stealth, thick}
    ]
        % Sensores
        \node[block, fill=green!20] (lidar) {LIDAR\\667 pontos};
        \node[block, fill=green!20, below=0.5cm of lidar] (camera) {Câmera RGB\\512×512};

        % Percepção
        \node[block, fill=yellow!20, right=1cm of lidar] (lidar_nn) {CNN 1D\\Obstáculos};
        \node[block, fill=yellow!20, right=1cm of camera] (camera_nn) {CNN\\Cores};

        % Decisão
        \node[block, fill=orange!20, right=1cm of lidar_nn, yshift=-0.75cm] (fuzzy) {Controlador\\Fuzzy};
        \node[block, fill=orange!20, below=0.5cm of fuzzy] (state) {Máquina de\\Estados};

        % Ação
        \node[block, fill=red!20, right=1cm of fuzzy] (base) {Base Móvel\\vx, vy, ω};
        \node[block, fill=red!20, below=0.5cm of base] (arm) {Braço +\\Garra};

        % Setas
        \draw[arrow] (lidar) -- (lidar_nn);
        \draw[arrow] (camera) -- (camera_nn);
        \draw[arrow] (lidar_nn) -- (fuzzy);
        \draw[arrow] (camera_nn) -- (fuzzy);
        \draw[arrow] (fuzzy) -- (base);
        \draw[arrow] (state) -- (arm);
        \draw[arrow] (fuzzy) -- (state);
    \end{tikzpicture}
    \end{center}

    \vspace{0.5em}
    \textbf{Frequência:} >10 Hz (ciclo completo em <100ms)
\end{frame}

\begin{frame}{Máquina de Estados}
    \begin{center}
    \begin{tikzpicture}[
        node distance=2cm,
        state/.style={circle, draw, fill=blue!20, minimum size=1.5cm, font=\small},
        arrow/.style={->, >=stealth, thick}
    ]
        \node[state] (search) {BUSCAR};
        \node[state, right=of search] (approach) {APROXIMAR};
        \node[state, right=of approach] (grasp) {PEGAR};
        \node[state, below=of grasp] (navigate) {NAVEGAR};
        \node[state, left=of navigate] (deposit) {DEPOSITAR};
        \node[state, fill=red!20, below=of search] (avoid) {DESVIAR};

        \draw[arrow] (search) -- node[above] {cubo} (approach);
        \draw[arrow] (approach) -- node[above] {perto} (grasp);
        \draw[arrow] (grasp) -- node[right] {ok} (navigate);
        \draw[arrow] (navigate) -- node[below] {caixa} (deposit);
        \draw[arrow] (deposit) -- node[left] {ok} (search);

        \draw[arrow, dashed, red] (search) -- node[left] {obstáculo} (avoid);
        \draw[arrow, dashed, red] (approach) -- (avoid);
        \draw[arrow, dashed, red] (avoid) -- node[below left] {livre} (search);
    \end{tikzpicture}
    \end{center}

    \textbf{DESVIAR:} Estado prioritário ativado por obstáculo <0.3m
\end{frame}

%=================================================
\section{Redes Neurais Artificiais}
%=================================================
\begin{frame}{RNA para LIDAR: Detecção de Obstáculos}
    \begin{columns}
        \begin{column}{0.5\textwidth}
            \textbf{Arquitetura Híbrida:}
            \begin{itemize}
                \item \textbf{Branch CNN 1D:}
                    \begin{itemize}
                        \item Input: 667 pontos normalizados
                        \item 3 camadas Conv1D + MaxPool
                        \item Output: 64 features
                    \end{itemize}
                \item \textbf{Branch Estatístico:}
                    \begin{itemize}
                        \item 6 features hand-crafted
                        \item min, max, mean, std, gradient
                    \end{itemize}
                \item \textbf{Fusão:} MLP (70→128→64→9)
            \end{itemize}
        \end{column}
        \begin{column}{0.5\textwidth}
            \textbf{Output: Mapa de 9 Setores}
            % TODO: Inserir diagrama de setores
            \begin{center}
                \textit{[Figura: 9 setores ao redor do robô]}

                \vspace{0.5em}
                Probabilidade de ocupação por setor
            \end{center}

            \vspace{1em}
            \textbf{Base Teórica:}
            \begin{itemize}
                \item Qi et al. (2017) - PointNet
                \item ~50K parâmetros
            \end{itemize}
        \end{column}
    \end{columns}
\end{frame}

\begin{frame}{RNA para Câmera: Classificação de Cores}
    \begin{columns}
        \begin{column}{0.5\textwidth}
            \textbf{Lightweight CNN:}
            \begin{itemize}
                \item Input: 512×512×3 RGB
                \item 3 blocos Conv2D + BN + ReLU + Pool
                \item Global Average Pooling
                \item FC: 128→64→3 (softmax)
            \end{itemize}

            \vspace{1em}
            \textbf{Características:}
            \begin{itemize}
                \item ~250K parâmetros
                \item >10 FPS em CPU
                \item Fallback: Segmentação HSV
            \end{itemize}
        \end{column}
        \begin{column}{0.5\textwidth}
            \textbf{Output: Classificação de Cor}
            % TODO: Inserir exemplos de detecção
            \begin{center}
                \textit{[Figura: Exemplos de detecção verde/azul/vermelho]}
            \end{center}

            \vspace{1em}
            \textbf{Base Teórica:}
            \begin{itemize}
                \item Goodfellow et al. (2016) - Deep Learning
                \item Redmon et al. (2016) - YOLO
            \end{itemize}
        \end{column}
    \end{columns}
\end{frame}

\begin{frame}{Segmentação HSV (Fallback)}
    \begin{columns}
        \begin{column}{0.5\textwidth}
            \textbf{Faixas Calibradas:}
            \begin{table}
                \footnotesize
                \begin{tabular}{|l|c|c|}
                    \hline
                    \textbf{Cor} & \textbf{H min-max} & \textbf{S/V min} \\
                    \hline
                    Verde & 35-85 & 100 \\
                    Azul & 100-130 & 100 \\
                    Vermelho & 0-10, 160-180 & 100 \\
                    \hline
                \end{tabular}
            \end{table}

            \vspace{1em}
            \textbf{Pipeline:}
            \begin{enumerate}
                \item RGB → HSV
                \item Threshold por cor
                \item Morfologia (open/close)
                \item Contornos → BBox
            \end{enumerate}
        \end{column}
        \begin{column}{0.5\textwidth}
            % TODO: Inserir pipeline visual
            \begin{center}
                \textit{[Figura: Pipeline de segmentação HSV]}

                \vspace{1em}
                Robusto a variações de iluminação
            \end{center}
        \end{column}
    \end{columns}
\end{frame}

%=================================================
\section{Lógica Fuzzy}
%=================================================
\begin{frame}{Controlador Fuzzy: Visão Geral}
    \textbf{Base Teórica:} Zadeh (1965), Mamdani \& Assilian (1975)

    \vspace{1em}
    \begin{columns}
        \begin{column}{0.5\textwidth}
            \textbf{Variáveis de Entrada:}
            \begin{itemize}
                \item distance\_to\_obstacle [0, 5]m
                \item angle\_to\_obstacle [-135, 135]°
                \item distance\_to\_cube [0, 3]m
                \item angle\_to\_cube [-135, 135]°
                \item cube\_detected (bool)
                \item holding\_cube (bool)
            \end{itemize}
        \end{column}
        \begin{column}{0.5\textwidth}
            \textbf{Variáveis de Saída:}
            \begin{itemize}
                \item linear\_velocity [-0.3, 0.3] m/s
                \item angular\_velocity [-0.5, 0.5] rad/s
                \item action (enum: search/approach/...)
            \end{itemize}

            \vspace{1em}
            \textbf{25 Regras} organizadas em:
            \begin{itemize}
                \item Segurança (15 regras)
                \item Busca (5 regras)
                \item Aproximação (5 regras)
            \end{itemize}
        \end{column}
    \end{columns}
\end{frame}

\begin{frame}{Funções de Pertinência}
    % TODO: Inserir gráficos das funções de pertinência
    \begin{columns}
        \begin{column}{0.5\textwidth}
            \textbf{distance\_to\_obstacle:}
            \begin{center}
                \textit{[Gráfico: MFs triangulares]}

                very\_near | near | medium | far | very\_far
            \end{center}

            \vspace{1em}
            \textbf{angle\_to\_obstacle:}
            \begin{center}
                \textit{[Gráfico: MFs triangulares]}

                neg\_big | neg\_med | ... | pos\_big
            \end{center}
        \end{column}
        \begin{column}{0.5\textwidth}
            \textbf{linear\_velocity:}
            \begin{center}
                \textit{[Gráfico: MFs de saída]}

                stop | slow | medium | fast
            \end{center}

            \vspace{1em}
            \textbf{Defuzzificação:}
            \begin{itemize}
                \item Método: Centróide
                \item Latência: <50ms
            \end{itemize}
        \end{column}
    \end{columns}
\end{frame}

\begin{frame}{Exemplos de Regras Fuzzy}
    \textbf{Regras de Segurança (Alta Prioridade):}
    \begin{itemize}
        \item \textbf{R001:} IF distance\_to\_obstacle IS very\_near AND angle\_to\_obstacle IS zero \\
              THEN linear\_velocity IS stop AND angular\_velocity IS turn\_right
        \item \textbf{R002:} IF distance\_to\_obstacle IS near AND angle\_to\_obstacle IS positive \\
              THEN linear\_velocity IS slow AND angular\_velocity IS turn\_left
    \end{itemize}

    \vspace{1em}
    \textbf{Regras de Aproximação:}
    \begin{itemize}
        \item \textbf{R016:} IF cube\_detected AND distance\_to\_cube IS far AND angle\_to\_cube IS zero \\
              THEN linear\_velocity IS fast AND angular\_velocity IS zero
        \item \textbf{R017:} IF cube\_detected AND distance\_to\_cube IS near AND angle\_to\_cube IS positive \\
              THEN linear\_velocity IS slow AND angular\_velocity IS turn\_left
    \end{itemize}
\end{frame}

\begin{frame}{Inferência e Defuzzificação}
    \begin{center}
    \begin{tikzpicture}[
        node distance=1.5cm,
        block/.style={rectangle, draw, fill=blue!20, text width=2.5cm, text centered, minimum height=1cm},
        arrow/.style={->, >=stealth, thick}
    ]
        \node[block, fill=green!20] (input) {Entradas\\Crisp};
        \node[block, fill=yellow!20, right=of input] (fuzz) {Fuzzificação};
        \node[block, fill=orange!20, right=of fuzz] (rules) {25 Regras\\Mamdani};
        \node[block, fill=red!20, right=of rules] (defuzz) {Centróide};
        \node[block, fill=purple!20, right=of defuzz] (output) {Saídas\\Crisp};

        \draw[arrow] (input) -- (fuzz);
        \draw[arrow] (fuzz) -- (rules);
        \draw[arrow] (rules) -- (defuzz);
        \draw[arrow] (defuzz) -- (output);
    \end{tikzpicture}
    \end{center}

    \vspace{1em}
    \textbf{Biblioteca:} scikit-fuzzy (Python)

    \textbf{Performance:} ~30ms por inferência (>30 Hz)
\end{frame}

%=================================================
\section{Demonstração}
%=================================================
\begin{frame}{Métricas de Performance}
    \begin{table}
        \begin{tabularx}{\textwidth}{|l|X|X|}
            \hline
            \textbf{Métrica} & \textbf{Meta} & \textbf{Resultado} \\
            \hline
            Cubos coletados & 15/15 & \textit{[preencher]} \\
            \hline
            Precisão de cor & >95\% & \textit{[preencher]} \\
            \hline
            Colisões & 0 & \textit{[preencher]} \\
            \hline
            Tempo total & <5 min & \textit{[preencher]} \\
            \hline
            FPS controle & >10 Hz & \textit{[preencher]} \\
            \hline
        \end{tabularx}
    \end{table}

    \vspace{1em}
    \textbf{Ambiente:}
    \begin{itemize}
        \item Webots R2023b
        \item Python 3.11 + PyTorch + scikit-fuzzy
        \item CPU Intel i7 (sem GPU)
    \end{itemize}
\end{frame}

\begin{frame}{Vídeo: Robô em Ação}
    \begin{center}
        % TODO: Inserir screenshots ou indicar vídeo
        \textit{[Demonstração ao vivo ou vídeo incorporado]}

        \vspace{2em}
        \Huge Demonstração

        \vspace{1em}
        \normalsize
        YouBot coletando e organizando cubos coloridos
    \end{center}
\end{frame}

%=================================================
\section{Conclusão}
%=================================================
\begin{frame}{Lições Aprendidas}
    \textbf{Desafios Superados:}
    \begin{itemize}
        \item Navegação sem GPS usando odometria + mapa local
        \item Integração RNA + Fuzzy em tempo real
        \item Calibração de parâmetros fuzzy para comportamento suave
    \end{itemize}

    \vspace{1em}
    \textbf{Trabalhos Futuros:}
    \begin{itemize}
        \item SLAM para mapeamento global
        \item Planejamento de trajetória com A*
        \item Treinamento end-to-end com reinforcement learning
    \end{itemize}
\end{frame}

\begin{frame}{Referências Principais}
    \footnotesize
    \begin{itemize}
        \item Goodfellow, I. et al. (2016). \textit{Deep Learning}. MIT Press.
        \item Zadeh, L.A. (1965). Fuzzy Sets. \textit{Information and Control}, 8(3), 338-353.
        \item Mamdani, E.H. \& Assilian, S. (1975). An experiment in linguistic synthesis with a fuzzy logic controller. \textit{International Journal of Man-Machine Studies}, 7(1), 1-13.
        \item Qi, C.R. et al. (2017). PointNet: Deep Learning on Point Sets. \textit{CVPR}.
        \item Thrun, S. et al. (2005). \textit{Probabilistic Robotics}. MIT Press.
        \item Bischoff, R. et al. (2011). The KUKA-DLR Lightweight Robot Arm. \textit{ICRA}.
    \end{itemize}
\end{frame}

\begin{frame}[plain]
    \begin{center}
        \Huge Obrigado!

        \vspace{2em}
        \normalsize
        \textbf{Luis Felipe Cordeiro Sena}

        \vspace{1em}
        MATA64 - Inteligência Artificial

        UFBA - Instituto de Computação

        \vspace{2em}
        \textit{Dúvidas?}
    \end{center}
\end{frame}

\end{document}
