\documentclass[aspectratio=169,xcolor=table]{beamer}
\usepackage[utf8]{inputenc}
\usepackage[T1]{fontenc}
\usepackage{lmodern}
\usepackage{csquotes}
\usepackage{xcolor}
\usepackage[portuguese]{babel}
\usepackage{hyperref}
\usepackage{tabularx}
\usepackage{amsmath}
\usepackage{amssymb}
\usepackage{tikz}
\usepackage[normalem]{ulem}  % For \sout{} strikethrough
\usetikzlibrary{shapes.geometric, arrows.meta, positioning, fit, calc, mindmap, shadows}

% ------------------------------------------------
% Tema e Configurações do Beamer
% ------------------------------------------------
\usetheme{DCC}

% Ajuste de espaçamento entre itens
\setbeamertemplate{itemize items}[circle]
\setbeamertemplate{itemize subitem}[circle]
\setlength{\itemsep}{0.6em}
\setlength{\parskip}{0.4em}

\graphicspath{{imgs/}{./imgs/}}

\author[Luis Felipe]{%
  \textbf{Luis Felipe Sena}
}
\title{Robô Autônomo para Coleta de Cubos}
\subtitle{Integração de Redes Neurais, Lógica Fuzzy e Planejamento A*}
\institute{Universidade Federal da Bahia \\ Instituto de Computação \\ MATA64 - Inteligência Artificial}
\date{Dezembro de 2025}

\begin{document}

%-------------------------------------------------
%  SLIDE DE TÍTULO
%-------------------------------------------------
\begin{frame}[plain,noframenumbering]
    \titlepage
\end{frame}

%-------------------------------------------------
%  SLIDE DE AGENDA
%-------------------------------------------------
\begin{frame}{Agenda}
    \begin{table}
        \begin{tabularx}{\textwidth}{|l|X|}
            \hline
            \textbf{Tempo} & \textbf{Conteúdo} \\
            \hline
            0--3 min  & \textbf{Problema e Desafios} -- Definição da tarefa e restrições \\
            \hline
            3--6 min  & \textbf{Percepção} -- LIDAR, Câmera RGB e Rede Neural (MobileNetV3) \\
            \hline
            6--9 min  & \textbf{Navegação Fuzzy} -- Regras linguísticas e funções de pertinência \\
            \hline
            9--12 min & \textbf{Planejamento A*} -- Grade de ocupação e busca de caminhos \\
            \hline
            12--15 min & \textbf{Arquitetura e Demo} -- Máquina de estados e demonstração \\
            \hline
        \end{tabularx}
    \end{table}
\end{frame}

\setlength{\parskip}{0.8em}

%=================================================
\section{Problema e Desafios}
%=================================================

\begin{frame}{O Problema}
    \begin{columns}[T]
        \begin{column}{0.55\textwidth}
            \begin{itemize}
                \item \textbf{Objetivo:} Coletar 15 cubos coloridos (vermelho, verde, azul)
                \item \textbf{Ação:} Depositar cada cubo na caixa de cor correspondente
                \item \textbf{Plataforma:} KUKA YouBot com rodas Mecanum (omnidirecional)
                \item \textbf{Ambiente:} Arena 7m $\times$ 4m com obstáculos fixos
            \end{itemize}
        \end{column}
        \begin{column}{0.42\textwidth}
            \centering
            % Placeholder for arena image
            \begin{tikzpicture}[scale=0.45]
                % Arena
                \draw[thick] (-3.5,-2) rectangle (3.5,2);
                % Robot
                \draw[fill=blue!30] (-2.5,0) rectangle (-2,0.4);
                \node[below] at (-2.25,-0.1) {\tiny YouBot};
                % Boxes
                \draw[fill=green!50] (2.8,1.3) rectangle (3.3,1.7);
                \draw[fill=blue!50] (2.8,-1.7) rectangle (3.3,-1.3);
                \draw[fill=red!50] (3.2,-0.2) rectangle (3.5,0.2);
                % Obstacles
                \foreach \x/\y in {0/0, 1.5/-1, 1.5/1, -1.8/1.2, -0.8/0.6, -0.8/-0.6, -1.8/-1.2} {
                    \draw[fill=brown!60] (\x-0.15,\y-0.15) rectangle (\x+0.15,\y+0.15);
                }
                % Cubes (small)
                \foreach \x/\y in {-1,0.5, 0.5,0.8, -0.3,-0.5, 1,-0.3} {
                    \draw[fill=red!80] (\x,\y) rectangle ++(0.08,0.08);
                }
                \foreach \x/\y in {-2,1, 0.2,0.3, -1.5,-0.8} {
                    \draw[fill=green!80] (\x,\y) rectangle ++(0.08,0.08);
                }
                \foreach \x/\y in {-0.5,1.2, 1.2,0.5, -1.2,-0.3} {
                    \draw[fill=blue!80] (\x,\y) rectangle ++(0.08,0.08);
                }
            \end{tikzpicture}
        \end{column}
    \end{columns}
\end{frame}

\begin{frame}{Restrições do Projeto}
    \centering
    \begin{tikzpicture}[scale=0.85]
        % Center: YouBot
        \node[draw, circle, fill=blue!30, minimum size=2.2cm, thick] (robot) at (0,0) {\small YouBot};

        % Required (green boxes around robot)
        \node[draw, rectangle, fill=green!30, rounded corners, text width=2.2cm, align=center, thick]
            (nn) at (-4,1.8) {\small Rede\\Neural};
        \node[draw, rectangle, fill=green!30, rounded corners, text width=2.2cm, align=center, thick]
            (fuzzy) at (4,1.8) {\small Lógica\\Fuzzy};
        \node[draw, rectangle, fill=green!30, rounded corners, text width=2.2cm, align=center, thick]
            (sensors) at (0,3.5) {\small LIDAR +\\Câmera};

        % Prohibited (red boxes, crossed out)
        \node[draw, rectangle, fill=red!30, rounded corners, text width=2cm, align=center, thick]
            (gps) at (-4,-1.8) {\small \sout{GPS}};
        \node[draw, rectangle, fill=red!30, rounded corners, text width=2cm, align=center, thick]
            (tele) at (0,-3) {\small \sout{Teleop}};
        \node[draw, rectangle, fill=red!30, rounded corners, text width=2cm, align=center, thick]
            (priv) at (4,-1.8) {\small \sout{Info Priv.}};

        % Arrows
        \draw[->, thick, green!70!black, line width=1.5pt] (nn) -- (robot);
        \draw[->, thick, green!70!black, line width=1.5pt] (fuzzy) -- (robot);
        \draw[->, thick, green!70!black, line width=1.5pt] (sensors) -- (robot);
        \draw[->, thick, red!70, dashed, line width=1.5pt] (gps) -- (robot);
        \draw[->, thick, red!70, dashed, line width=1.5pt] (tele) -- (robot);
        \draw[->, thick, red!70, dashed, line width=1.5pt] (priv) -- (robot);

        % Legend
        \node[below] at (0,-4.2) {\small \textcolor{green!70!black}{Verde = Obrigatório} \quad \textcolor{red!70}{Vermelho = Proibido}};
    \end{tikzpicture}
\end{frame}

%=================================================
\section{Sistema de Percepção}
%=================================================

\begin{frame}{Sensores do YouBot}
    \centering
    \begin{tikzpicture}[scale=0.75]
        % Robot body
        \draw[fill=gray!30, thick, rounded corners=3pt] (-1.3,-1.8) rectangle (1.3,1.8);
        \node at (0,0) {YouBot};

        % LIDAR rays
        \foreach \a in {-90,-72,...,90} {
            \draw[red!60, ->, line width=0.7pt] (0,1.8) -- ++(\a:2.5);
        }
        \node[red, font=\small\bfseries] at (0,4.6) {LIDAR 180°};
        \node[red, font=\scriptsize] at (0,4.1) {(0.1--5.0m)};

        % Camera FOV
        \draw[blue!60, fill=blue!15, line width=1pt] (0,1.8) -- ++(55:2) arc (55:125:2) -- cycle;
        \node[blue, font=\small\bfseries] at (0,3.3) {Câmera RGB};

        % Distance sensors (front + rear)
        \draw[green!70!black, thick, ->] (-0.8,1.8) -- ++(0,0.7);
        \draw[green!70!black, thick, ->] (0,1.8) -- ++(0,0.7);
        \draw[green!70!black, thick, ->] (0.8,1.8) -- ++(0,0.7);
        \draw[green!70!black, thick, ->] (-0.8,-1.8) -- ++(0,-0.7);
        \draw[green!70!black, thick, ->] (0,-1.8) -- ++(0,-0.7);
        \draw[green!70!black, thick, ->] (0.8,-1.8) -- ++(0,-0.7);
        \node[green!70!black, font=\small\bfseries] at (0,-3) {6 Dist. Sensors};

        % Legend box
        \node[draw, rounded corners, fill=yellow!10, font=\scriptsize, text width=5cm, align=center] at (5,0) {
            \textbf{Função}\\[0.2em]
            \textcolor{red!70}{LIDAR} $\rightarrow$ Obstáculos\\
            \textcolor{blue!70}{Câmera} $\rightarrow$ Cubos + CNN\\
            \textcolor{green!60!black}{Dist.} $\rightarrow$ Colisão
        };
    \end{tikzpicture}
\end{frame}

\begin{frame}{Rede Neural -- MobileNetV3-Small}
    \begin{columns}[T]
        \begin{column}{0.55\textwidth}
            \textbf{Arquitetura CNN}
            \begin{itemize}
                \item Backbone: MobileNetV3-Small (pré-treinado ImageNet)
                \item Classificador customizado: 256 $\rightarrow$ 3 classes
                \item Entrada: 64$\times$64 RGB normalizada
                \item Saída: Probabilidades (vermelho/verde/azul)
            \end{itemize}

            \vspace{0.5em}
            \textbf{Treinamento (Transfer Learning)}
            \begin{itemize}
                \item Fase 1: Congelar backbone, treinar apenas classificador
                \item Fase 2: Fine-tuning com learning rate reduzido
                \item Acurácia validação: \textbf{99.4\%}
            \end{itemize}
        \end{column}
        \begin{column}{0.42\textwidth}
            \centering
            % CNN diagram
            \begin{tikzpicture}[scale=0.6, every node/.style={scale=0.8}]
                % Input
                \draw[fill=blue!20] (0,0) rectangle (1,2);
                \node[below] at (0.5,-0.2) {\tiny 64x64x3};

                % Conv layers
                \draw[fill=orange!30] (2,0.2) rectangle (2.8,1.8);
                \draw[fill=orange!30] (3.2,0.4) rectangle (3.8,1.6);
                \draw[fill=orange!30] (4.2,0.5) rectangle (4.6,1.5);
                \node[below] at (3.4,-0.2) {\tiny MobileNetV3};

                % FC layers
                \draw[fill=green!30] (5.5,0.6) rectangle (6,1.4);
                \draw[fill=green!30] (6.5,0.7) rectangle (6.8,1.3);

                % Output
                \draw[fill=red!30] (7.5,0.7) rectangle (8,1.3);
                \node[below] at (7.75,-0.2) {\tiny 3 classes};

                % Arrows
                \draw[->, thick] (1.1,1) -- (1.9,1);
                \draw[->, thick] (4.7,1) -- (5.4,1);
                \draw[->, thick] (6.9,1) -- (7.4,1);
            \end{tikzpicture}

            \vspace{1em}
            \begin{tikzpicture}
                \node[draw, rounded corners, fill=green!20, text width=4cm, align=center, font=\small] {
                    Fallback: Heurísticas HSV quando confiança $<$ 0.5
                };
            \end{tikzpicture}
        \end{column}
    \end{columns}
\end{frame}

%=================================================
\section{Navegação Fuzzy}
%=================================================

\begin{frame}{Lógica Fuzzy -- Conceito}
    \centering
    \begin{columns}[T]
        \begin{column}{0.48\textwidth}
            \centering
            % Membership function diagram (larger)
            \begin{tikzpicture}[scale=0.95]
                % Axes
                \draw[->, thick] (0,0) -- (5.5,0) node[right] {dist(m)};
                \draw[->, thick] (0,0) -- (0,2.8) node[above] {$\mu$};

                % muito_perto
                \draw[line width=2pt, red] (0,2.3) -- (0.5,2.3) -- (1.5,0);
                \node[red, font=\scriptsize] at (0.7,2.6) {muito\_perto};

                % perto
                \draw[line width=2pt, orange] (0.5,0) -- (1.5,2.3) -- (2.5,0);
                \node[orange, font=\scriptsize] at (1.5,2.6) {perto};

                % longe
                \draw[line width=2pt, green!70!black] (2,0) -- (3,2.3) -- (5.2,2.3);
                \node[green!70!black, font=\scriptsize] at (4.2,2.6) {longe};

                % Ticks
                \foreach \x/\l in {0.8/0.25, 1.7/0.45, 3.2/1.0} {
                    \draw (\x,0) -- (\x,-0.15) node[below, font=\tiny] {\l m};
                }
            \end{tikzpicture}

            \vspace{0.3em}
            {\footnotesize Funções de pertinência}
        \end{column}
        \begin{column}{0.48\textwidth}
            \centering
            % Why Fuzzy - visual
            \begin{tikzpicture}[scale=0.8]
                % Crisp vs Fuzzy comparison
                \node[draw, rectangle, fill=gray!20, minimum width=3cm, minimum height=1cm] (crisp) at (0,2) {Crisp};
                \node[draw, rectangle, fill=blue!20, minimum width=3cm, minimum height=1cm, rounded corners] (fuzzy) at (0,0) {Fuzzy};

                % Labels
                \node[right, font=\scriptsize] at (1.8,2) {0 ou 1};
                \node[right, font=\scriptsize] at (1.8,0) {0.0 -- 1.0};

                % Arrow
                \draw[->, thick, green!60!black] (0,1.3) -- (0,0.7) node[midway, right, font=\scriptsize] {suave};
            \end{tikzpicture}

            \vspace{0.5em}
            \footnotesize
            \textbf{Vantagens:}\\
            $\bullet$ Transições suaves\\
            $\bullet$ Lida com incerteza\\
            $\bullet$ Regras intuitivas
        \end{column}
    \end{columns}
\end{frame}

\begin{frame}{Regras Fuzzy de Navegação}
    \centering
    \begin{tikzpicture}[scale=0.75]
        % Scenario 1: Obstacle front - Reverse + Strafe
        \begin{scope}[shift={(0,0)}]
            \draw[fill=blue!20, thick] (0,0) rectangle (0.8,1.4);
            \node[font=\tiny] at (0.4,0.7) {R};
            \draw[fill=red!50, thick] (0.2,1.8) rectangle (0.6,2.2);
            \draw[->, ultra thick, red!70] (0.4,0) -- (0.4,-0.7);
            \draw[->, thick, green!60!black] (0.8,0.7) -- (1.4,0.7);
            \node[below, font=\scriptsize] at (0.4,-1) {Ré + Strafe};
        \end{scope}

        % Scenario 2: Obstacle left - Strafe right
        \begin{scope}[shift={(4,0)}]
            \draw[fill=blue!20, thick] (0,0) rectangle (0.8,1.4);
            \node[font=\tiny] at (0.4,0.7) {R};
            \draw[fill=orange!50, thick] (-0.6,0.4) rectangle (-0.2,1);
            \draw[->, thick, green!60!black] (0.8,0.7) -- (1.4,0.7);
            \node[below, font=\scriptsize] at (0.4,-1) {Strafe direita};
        \end{scope}

        % Scenario 3: Aligned with target - Forward
        \begin{scope}[shift={(8,0)}]
            \draw[fill=blue!20, thick] (0,0) rectangle (0.8,1.4);
            \node[font=\tiny] at (0.4,0.7) {R};
            \draw[fill=green!50, thick] (0.2,2.5) rectangle (0.6,2.9);
            \draw[->, ultra thick, green!60!black] (0.4,1.4) -- (0.4,2.3);
            \node[below, font=\scriptsize] at (0.4,-1) {Avançar};
        \end{scope}

        % Scenario 4: Angle big - Rotate only
        \begin{scope}[shift={(12,0)}]
            \draw[fill=blue!20, thick] (0,0) rectangle (0.8,1.4);
            \node[font=\tiny] at (0.4,0.7) {R};
            \draw[fill=green!50, thick] (1.8,2) rectangle (2.2,2.4);
            \draw[->, thick, orange, rotate around={30:(0.4,0.7)}] (0.4,1.4) -- (0.4,2);
            \node[below, font=\scriptsize] at (0.4,-1) {Rotacionar};
        \end{scope}
    \end{tikzpicture}

    \vspace{0.5em}

    % Rules table (compact)
    \footnotesize
    \begin{tabular}{|c|c|}
    \hline
    \textbf{SE} & \textbf{ENTÃO} \\
    \hline
    obstáculo muito perto & reverso + strafe \\
    lateral bloqueado & strafe oposto \\
    ângulo pequeno & avançar rápido \\
    \hline
    \end{tabular}
\end{frame}

%=================================================
\section{Planejamento A*}
%=================================================

\begin{frame}{Algoritmo A* -- Visão Geral}
    \begin{columns}[T]
        \begin{column}{0.5\textwidth}
            \textbf{Função de Custo}
            \[ f(n) = g(n) + h(n) \]
            \begin{itemize}
                \item $g(n)$: Custo do início até $n$
                \item $h(n)$: Heurística (Manhattan) até objetivo
                \item $f(n)$: Custo total estimado
            \end{itemize}

            \vspace{0.5em}
            \textbf{Grade de Ocupação}
            \begin{itemize}
                \item Célula: 12cm $\times$ 12cm
                \item Arena: 58 $\times$ 33 células
                \item Atualizada por raycasting (LIDAR)
            \end{itemize}
        \end{column}
        \begin{column}{0.45\textwidth}
            \centering
            % A* grid visualization
            \begin{tikzpicture}[scale=0.5]
                % Grid
                \foreach \x in {0,1,...,8} {
                    \draw[gray!30] (\x,0) -- (\x,6);
                }
                \foreach \y in {0,1,...,6} {
                    \draw[gray!30] (0,\y) -- (8,\y);
                }

                % Obstacles (inflated)
                \fill[brown!60] (3,2) rectangle (5,4);

                % Start
                \fill[green!70] (1,1) rectangle (2,2);
                \node at (1.5,1.5) {\tiny S};

                % Goal
                \fill[red!70] (7,5) rectangle (8,6);
                \node at (7.5,5.5) {\tiny G};

                % Path
                \draw[blue!70, line width=2pt, ->] (1.5,1.5) -- (1.5,2.5) -- (2.5,3.5) -- (2.5,4.5) -- (3.5,5.5) -- (5.5,5.5) -- (7.5,5.5);

                % Legend
                \node[below] at (4,-0.5) {\tiny Caminho A*};
            \end{tikzpicture}
        \end{column}
    \end{columns}
\end{frame}

\begin{frame}{Inflação de Obstáculos}
    \begin{columns}[T]
        \begin{column}{0.5\textwidth}
            \textbf{Por que inflar?}
            \begin{itemize}
                \item Robô não é um ponto
                \item YouBot: 58cm $\times$ 38cm
                \item Margem de segurança: \textbf{30cm}
            \end{itemize}

            \vspace{0.5em}
            \textbf{Obstáculos Conhecidos}
            \begin{itemize}
                \item 7 caixas de madeira (fixas)
                \item 3 caixas de depósito (destinos)
                \item Paredes da arena
            \end{itemize}
        \end{column}
        \begin{column}{0.45\textwidth}
            \centering
            % Inflation visualization
            \begin{tikzpicture}[scale=0.8]
                % Original obstacle
                \fill[brown!70] (0,0) rectangle (1,1);
                \node at (0.5,0.5) {\tiny Obs};

                % Inflated area
                \draw[red!50, dashed, thick] (-0.5,-0.5) rectangle (1.5,1.5);

                % Robot
                \draw[blue!50, fill=blue!20] (2.2,0.2) rectangle (2.8,0.8);
                \node[right] at (2.9,0.5) {\tiny Robô};

                % Arrow showing safe path
                \draw[->, thick, green!70] (2.5,-0.5) -- (2.5,1.5);

                % Labels
                \node[below] at (0.5,-0.8) {\small Real};
                \node[below, red] at (0.5,-1.3) {\small Inflado};
            \end{tikzpicture}

            \vspace{0.5em}
            {\small A inflação garante que o centro do robô nunca colida}
        \end{column}
    \end{columns}
\end{frame}

%=================================================
\section{Arquitetura do Sistema}
%=================================================

\begin{frame}{Máquina de Estados Finita}
    \centering
    \begin{tikzpicture}[
        scale=0.9,
        state/.style={draw, rectangle, rounded corners, minimum width=2cm, minimum height=0.9cm, fill=blue!20, font=\small},
        arrow/.style={->, thick, >=stealth}
    ]
        % States
        \node[state, fill=yellow!30] (search) at (0,0) {SEARCH};
        \node[state, fill=orange!30] (approach) at (4,0) {APPROACH};
        \node[state, fill=green!30] (grasp) at (8,0) {GRASP};
        \node[state, fill=purple!30] (tobox) at (8,-2.5) {TO\_BOX};
        \node[state, fill=red!30] (drop) at (4,-2.5) {DROP};

        % Transitions
        \draw[arrow] (search) -- node[above, font=\scriptsize] {detectado} (approach);
        \draw[arrow] (approach) -- node[above, font=\scriptsize] {perto} (grasp);
        \draw[arrow] (grasp) -- node[right, font=\scriptsize] {agarrado} (tobox);
        \draw[arrow] (tobox) -- node[above, font=\scriptsize] {chegou} (drop);
        \draw[arrow] (drop) -- node[left, font=\scriptsize] {ok} (search);

        % Self-loop
        \draw[arrow] (approach.north) .. controls (2,1.3) and (-2,1.3) .. node[above, font=\scriptsize] {perdido} (search.north);
    \end{tikzpicture}
\end{frame}

\begin{frame}{Hierarquia de Prioridades}
    \centering
    \begin{columns}[c]
        \begin{column}{0.45\textwidth}
            \centering
            % Priority pyramid (larger)
            \begin{tikzpicture}[scale=1.0]
                % Pyramid layers
                \fill[red!70] (0,0) -- (2.5,0) -- (1.25,1.2) -- cycle;
                \fill[orange!60] (-0.6,-1.2) -- (3.1,-1.2) -- (2.5,0) -- (0,0) -- cycle;
                \fill[yellow!50] (-1.2,-2.4) -- (3.7,-2.4) -- (3.1,-1.2) -- (-0.6,-1.2) -- cycle;

                % Labels
                \node[font=\small\bfseries, white] at (1.25,0.4) {Paredes};
                \node[font=\small\bfseries] at (1.25,-0.6) {Obstáculos};
                \node[font=\small\bfseries] at (1.25,-1.8) {Cubos};

                % Priority arrow
                \draw[->, ultra thick] (4.5,1) -- (4.5,-2.2);
                \node[right, font=\scriptsize, rotate=-90] at (4.8,-0.6) {Prioridade};
            \end{tikzpicture}
        \end{column}
        \begin{column}{0.5\textwidth}
            \footnotesize
            \textbf{Segurança:}\\
            1. Nunca colidir com paredes\\
            2. Evitar empurrar cubos\\
            3. Manter progresso

            \vspace{0.5em}
            \textbf{Recuperação (stuck $>$ 3s):}\\
            $\rightarrow$ Escape para corredor seguro
        \end{column}
    \end{columns}
\end{frame}

\begin{frame}{Integração dos Componentes}
    \centering
    \begin{tikzpicture}[
        scale=0.7,
        box/.style={draw, rectangle, rounded corners, minimum width=2.5cm, minimum height=0.8cm},
        arrow/.style={->, thick}
    ]
        % Sensors
        \node[box, fill=blue!20] (lidar) at (0,3) {LIDAR};
        \node[box, fill=blue!20] (camera) at (0,1.5) {Câmera};
        \node[box, fill=blue!20] (dist) at (0,0) {Dist. Sensors};

        % Processing
        \node[box, fill=orange!20] (grid) at (4,3) {Grade Ocupação};
        \node[box, fill=orange!20] (cnn) at (4,1.5) {CNN (cor)};
        \node[box, fill=orange!20] (fuzzy) at (4,0) {Fuzzy Nav};

        % Planning
        \node[box, fill=green!20] (astar) at (8,2.25) {A* Planner};
        \node[box, fill=green!20] (fsm) at (8,0.75) {FSM};

        % Output
        \node[box, fill=red!20] (motor) at (12,1.5) {Motores};

        % Arrows
        \draw[arrow] (lidar) -- (grid);
        \draw[arrow] (camera) -- (cnn);
        \draw[arrow] (dist) -- (fuzzy);
        \draw[arrow] (grid) -- (astar);
        \draw[arrow] (cnn) -- (fsm);
        \draw[arrow] (fuzzy) -- (fsm);
        \draw[arrow] (astar) -- (fsm);
        \draw[arrow] (fsm) -- (motor);
    \end{tikzpicture}

    \vspace{1em}
    {\small Pipeline completo: Sensores $\rightarrow$ Processamento $\rightarrow$ Planejamento $\rightarrow$ Atuação}
\end{frame}

%=================================================
\section{Resultados e Demo}
%=================================================

\begin{frame}{Sequência de Coleta}
    \centering
    \begin{tikzpicture}[scale=0.65]
        % Step 1: Search
        \begin{scope}[shift={(0,0)}]
            \draw[thick, gray] (-1.5,-1) rectangle (1.5,1.5);
            \draw[fill=blue!30] (-0.3,-0.5) rectangle (0.3,0.1);
            \draw[red!60, dashed] (0,0.1) -- ++(30:1.5);
            \draw[red!60, dashed] (0,0.1) -- ++(-30:1.5);
            \draw[fill=red!70] (0.8,0.8) rectangle (1,1);
            \node[below, font=\scriptsize\bfseries] at (0,-1.3) {1. SEARCH};
            \draw[->, thick, blue!60] (-0.5,-0.2) arc (180:90:0.3);
        \end{scope}

        % Arrow
        \draw[->, ultra thick, green!60!black] (2,0.3) -- (3,0.3);

        % Step 2: Approach
        \begin{scope}[shift={(5,0)}]
            \draw[thick, gray] (-1.5,-1) rectangle (1.5,1.5);
            \draw[fill=blue!30] (-0.3,-0.3) rectangle (0.3,0.3);
            \draw[fill=red!70] (0.6,0.5) rectangle (0.8,0.7);
            \draw[->, thick, green!60!black] (0,0.3) -- (0.5,0.5);
            \node[below, font=\scriptsize\bfseries] at (0,-1.3) {2. APPROACH};
        \end{scope}

        % Arrow
        \draw[->, ultra thick, green!60!black] (7,0.3) -- (8,0.3);

        % Step 3: Grasp
        \begin{scope}[shift={(10,0)}]
            \draw[thick, gray] (-1.5,-1) rectangle (1.5,1.5);
            \draw[fill=blue!30] (-0.3,-0.1) rectangle (0.3,0.5);
            % Arm extended
            \draw[fill=gray!50, thick] (0,0.5) rectangle (0.15,1.1);
            % Gripper
            \draw[fill=orange!60, thick] (-0.1,1.1) rectangle (0.25,1.25);
            \draw[fill=red!70] (0,0.85) rectangle (0.15,1);
            \node[below, font=\scriptsize\bfseries] at (0,-1.3) {3. GRASP};
        \end{scope}
    \end{tikzpicture}

    \vspace{0.8em}

    \begin{tikzpicture}[scale=0.65]
        % Step 4: TO_BOX
        \begin{scope}[shift={(0,0)}]
            \draw[thick, gray] (-1.8,-1.2) rectangle (1.8,1.5);
            \draw[fill=blue!30] (-1.2,0) rectangle (-0.6,0.6);
            % Cube on robot
            \draw[fill=red!70] (-1,0.6) rectangle (-0.8,0.8);
            % Target box
            \draw[fill=red!40, thick] (1,0.2) rectangle (1.5,0.7);
            % A* path
            \draw[blue!60, thick, dashed, ->] (-0.9,0.3) -- (-0.2,0.3) -- (0.3,0.6) -- (0.9,0.45);
            \node[below, font=\scriptsize\bfseries] at (0,-1.5) {4. TO\_BOX (A*)};
        \end{scope}

        % Arrow
        \draw[->, ultra thick, green!60!black] (2.3,0.3) -- (3.3,0.3);

        % Step 5: Drop
        \begin{scope}[shift={(5.5,0)}]
            \draw[thick, gray] (-1.5,-1.2) rectangle (1.5,1.5);
            \draw[fill=blue!30] (-0.3,0) rectangle (0.3,0.6);
            % Box
            \draw[fill=red!40, thick] (0.5,-0.3) rectangle (1.2,0.5);
            % Cube dropping
            \draw[fill=red!70] (0.7,0.1) rectangle (0.9,0.3);
            \draw[->, thick, orange] (0.1,0.6) -- (0.65,0.2);
            \node[below, font=\scriptsize\bfseries] at (0,-1.5) {5. DROP};
        \end{scope}

        % Arrow (loop back)
        \draw[->, ultra thick, green!60!black] (7.5,0.3) -- (8.5,0.3);

        % Step 6: Repeat
        \begin{scope}[shift={(10.5,0)}]
            \draw[thick, gray] (-1.5,-1.2) rectangle (1.5,1.5);
            \draw[fill=blue!30] (-0.3,-0.3) rectangle (0.3,0.3);
            \draw[->, thick, blue!60] (0,0.3) arc (90:450:0.4);
            \node[font=\scriptsize] at (0,0) {\tiny $\times$15};
            \node[below, font=\scriptsize\bfseries] at (0,-1.5) {6. REPETIR};
        \end{scope}
    \end{tikzpicture}
\end{frame}

\begin{frame}{Visualização LIDAR + Câmera}
    \centering
    \begin{tikzpicture}[scale=0.8]
        % Arena
        \draw[thick] (-4,-2.5) rectangle (4,2.5);

        % Robot at center
        \draw[fill=blue!40, thick] (-0.4,-0.25) rectangle (0.4,0.25);
        \node[font=\tiny] at (0,0) {YouBot};

        % LIDAR rays (180 degrees)
        \foreach \a in {-85,-70,-55,-40,-25,-10,5,20,35,50,65,80} {
            \pgfmathsetmacro{\len}{1.5+rand*0.5}
            \draw[red!50, thick] (0,0.25) -- ++(\a+90:\len);
        }

        % Obstacle detected by LIDAR
        \draw[fill=brown!60] (1.2,1) rectangle (1.8,1.6);
        \draw[red!80, ultra thick] (0,0.25) -- (1.3,1);

        % Camera FOV cone
        \draw[blue!40, fill=blue!10, thick] (0,0.25) -- ++(60:2.5) arc (60:120:2.5) -- cycle;

        % Cube in camera view
        \draw[fill=green!70] (-0.8,1.8) rectangle (-0.5,2.1);
        \node[green!70!black, font=\scriptsize] at (-0.65,2.4) {Detectado!};

        % Distance to cube
        \draw[<->, thick, green!60!black] (0,0.25) -- (-0.65,1.8);
        \node[green!60!black, font=\tiny, right] at (-0.3,1) {1.6m};

        % Legend
        \node[draw, fill=white, rounded corners, font=\scriptsize, text width=2.8cm, align=left] at (-2.8,-1.5) {
            \textcolor{red!70}{---} LIDAR\\[0.1em]
            \textcolor{blue!50}{$\triangle$} Câmera FOV\\[0.1em]
            \textcolor{green!60!black}{$\bullet$} Cubo detectado
        };

        % Recognition box
        \node[draw, fill=yellow!20, rounded corners, font=\scriptsize, text width=1.8cm, align=center] at (2.5,-1.5) {
            CNN: Verde\\[0.1em]
            Conf: 0.97
        };
    \end{tikzpicture}
\end{frame}

\begin{frame}{Navegação A* em Ação}
    \centering
    \begin{tikzpicture}[scale=0.55]
        % Arena grid
        \draw[step=0.5, gray!20, thin] (-4,-2.5) grid (4,2.5);
        \draw[thick] (-4,-2.5) rectangle (4,2.5);

        % Obstacles (wooden boxes) with inflation shown
        \foreach \x/\y in {0.5/0, 2/-1.2, 2/1.2, -2.3/1.5, -1/-0.7, -1/0.7, -2.3/-1.5} {
            \fill[brown!40, opacity=0.5] (\x-0.4,\y-0.4) rectangle (\x+0.4,\y+0.4);
            \fill[brown!70] (\x-0.2,\y-0.2) rectangle (\x+0.2,\y+0.2);
        }

        % Deposit boxes
        \draw[fill=green!50, thick] (3.2,1.3) rectangle (3.8,1.8);
        \node[font=\tiny] at (3.5,1.55) {G};
        \draw[fill=blue!50, thick] (3.2,-1.8) rectangle (3.8,-1.3);
        \node[font=\tiny] at (3.5,-1.55) {B};
        \draw[fill=red!50, thick] (3.5,-0.25) rectangle (4,0.25);
        \node[font=\tiny] at (3.75,0) {R};

        % Robot position
        \draw[fill=blue!60, thick] (-3.2,-0.2) rectangle (-2.6,0.2);
        \node[font=\tiny, white] at (-2.9,0) {R};

        % A* path to red box (avoiding obstacles)
        \draw[blue!70, line width=2.5pt, ->]
            (-2.6,0) -- (-1.5,0) -- (-1.5,1.5) -- (0.8,1.5) -- (0.8,0.5) -- (1.5,0.5) -- (2.5,0) -- (3.5,0);

        % Waypoints
        \foreach \x/\y in {-1.5/0, -1.5/1.5, 0.8/1.5, 0.8/0.5, 1.5/0.5, 2.5/0} {
            \fill[blue!70] (\x,\y) circle (0.1);
        }

        % Legend
        \node[draw, fill=white, rounded corners, font=\scriptsize, text width=2cm, align=left] at (-2.8,-2) {
            \textcolor{brown!70}{$\blacksquare$} Obstáculo\\[0.1em]
            \textcolor{brown!40}{$\square$} Inflação\\[0.1em]
            \textcolor{blue!70}{---} A* path
        };

        % Label
        \node[font=\small\bfseries] at (0,3) {Caminho calculado evitando obstáculos};
    \end{tikzpicture}
\end{frame}

\begin{frame}{Resultados}
    \begin{columns}[T]
        \begin{column}{0.5\textwidth}
            \textbf{Métricas da Rede Neural}
            \begin{itemize}
                \item Acurácia: 99.4\%
                \item Formato: ONNX (portável)
                \item Inferência: $<$ 10ms por frame
            \end{itemize}

            \vspace{0.5em}
            \textbf{Navegação}
            \begin{itemize}
                \item A* encontra caminhos válidos consistentemente
                \item Fuzzy evita colisões em tempo real
                \item Recuperação de travamento funcional
            \end{itemize}
        \end{column}
        \begin{column}{0.45\textwidth}
            \textbf{Desafios Encontrados}
            \begin{itemize}
                \item Drift de odometria ao longo do tempo
                \item Cubos empurrados bloqueando caminhos
                \item Coordenação braço-navegação
            \end{itemize}

            \vspace{0.5em}
            \textbf{Soluções}
            \begin{itemize}
                \item Sincronização periódica com ground truth
                \item Rotas via corredores seguros
                \item Máquina de estados bem definida
            \end{itemize}
        \end{column}
    \end{columns}
\end{frame}

\begin{frame}{Demonstração}
    \centering
    \vspace{2em}

    \begin{tikzpicture}
        \node[draw, rounded corners, fill=blue!10, text width=12cm, align=center, minimum height=5cm] {
            \Huge \textbf{DEMO}

            \vspace{1em}
            \large Execução do robô coletando cubos no Webots
        };
    \end{tikzpicture}
\end{frame}

%=================================================
\section{Conclusão}
%=================================================

\begin{frame}{Conclusão}
    \begin{columns}[T]
        \begin{column}{0.48\textwidth}
            \textbf{Contribuições}
            \begin{itemize}
                \item Integração bem-sucedida de CNN + Fuzzy + A*
                \item Sistema de navegação robusto e reativo
                \item Arquitetura modular e extensível
            \end{itemize}
        \end{column}
        \begin{column}{0.48\textwidth}
            \textbf{Trabalhos Futuros}
            \begin{itemize}
                \item SLAM para melhor localização
                \item Planejamento com obstáculos dinâmicos
                \item Otimização da ordem de coleta
            \end{itemize}
        \end{column}
    \end{columns}

    \vspace{1.5em}
    \centering
    \begin{tikzpicture}
        \node[draw, rounded corners, fill=green!20, text width=10cm, align=center] {
            \large \textbf{Obrigado!}

            \vspace{0.3em}
            Perguntas?
        };
    \end{tikzpicture}
\end{frame}

%-------------------------------------------------
%  SLIDE DE REFERÊNCIAS
%-------------------------------------------------
\begin{frame}{Referências}
    \small
    \begin{itemize}
        \item Hart, P. E., Nilsson, N. J., \& Raphael, B. (1968). \textit{A Formal Basis for the Heuristic Determination of Minimum Cost Paths}
        \item Zadeh, L. A. (1965). \textit{Fuzzy Sets} -- Information and Control
        \item Howard, A., et al. (2019). \textit{Searching for MobileNetV3}
        \item Elfes, A. (1989). \textit{Using Occupancy Grids for Mobile Robot Perception and Navigation}
    \end{itemize}
\end{frame}

\end{document}
