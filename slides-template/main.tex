\documentclass[aspectratio=169,xcolor=table]{beamer}
\usepackage[utf8]{inputenc}
\usepackage[T1]{fontenc}
\usepackage{lmodern}
\usepackage{csquotes}
\usepackage{xcolor}
\usepackage[portuguese]{babel}
\usepackage{hyperref}
\usepackage{tabularx}
\usepackage{amsmath}
\usepackage{amssymb}
\usepackage{tikz}
\usepackage[normalem]{ulem}  % For \sout{} strikethrough
\usetikzlibrary{shapes.geometric, arrows.meta, positioning, fit, calc, mindmap, shadows}

% ------------------------------------------------
% Tema e Configurações do Beamer
% ------------------------------------------------
\usetheme{DCC}

% Ajuste de espaçamento entre itens
\setbeamertemplate{itemize items}[circle]
\setbeamertemplate{itemize subitem}[circle]
\setlength{\itemsep}{0.6em}
\setlength{\parskip}{0.4em}

\graphicspath{{imgs/}{./imgs/}}

\author[Luis Felipe]{%
  \textbf{Luis Felipe Sena}
}
\title{Robô Autônomo para Coleta de Cubos}
\subtitle{Integração de Redes Neurais, Lógica Fuzzy e Planejamento A*}
\institute{Universidade Federal da Bahia \\ Instituto de Computação \\ MATA64 - Inteligência Artificial}
\date{Dezembro de 2025}

\begin{document}

%-------------------------------------------------
%  SLIDE DE TÍTULO
%-------------------------------------------------
\begin{frame}[plain,noframenumbering]
    \titlepage
\end{frame}

%-------------------------------------------------
%  SLIDE DE AGENDA (visual)
%-------------------------------------------------
\begin{frame}{Agenda}
    \centering
    \begin{tikzpicture}[scale=0.9]
        % Timeline visual
        \foreach \i/\t/\c/\lab in {
            0/0--3min/blue!60/Problema,
            1/3--6min/orange!60/Percepção,
            2/6--9min/green!60/Fuzzy,
            3/9--12min/purple!60/A*,
            4/12--15min/red!60/Demo%
        } {
            \node[draw, rectangle, rounded corners, fill=\c, minimum width=2.2cm, minimum height=1.2cm, font=\small\bfseries] 
                at (\i*2.8, 0) {\lab};
            \node[font=\tiny, below] at (\i*2.8, -0.9) {\t};
        }
        % Arrows
        \foreach \i in {0,...,3} {
            \pgfmathsetmacro{\next}{\i+1}
            \draw[->, thick, gray] (\i*2.8+1.15, 0) -- (\next*2.8-1.15, 0);
        }
    \end{tikzpicture}
\end{frame}

\setlength{\parskip}{0.8em}

%=================================================
\section{Problema e Desafios}
%=================================================

\begin{frame}{O Problema}
    \centering
    \begin{tikzpicture}[scale=0.65]
        % Arena grande
        \draw[thick, fill=gray!10] (-4.5,-2.5) rectangle (4.5,2.5);
        \node[font=\tiny] at (-3.8,2.2) {7m × 4m};
        
        % Robot
        \draw[fill=blue!40, thick] (-3.5,-0.2) rectangle (-2.8,0.4);
        \node[below, font=\tiny\bfseries] at (-3.15,-0.3) {YouBot};
        
        % Deposit boxes (large, prominent)
        \draw[fill=green!60, thick] (3.5,1.4) rectangle (4.2,2.1);
        \node[font=\scriptsize\bfseries, white] at (3.85,1.75) {G};
        \draw[fill=blue!60, thick] (3.5,-2.1) rectangle (4.2,-1.4);
        \node[font=\scriptsize\bfseries, white] at (3.85,-1.75) {B};
        \draw[fill=red!60, thick] (4.0,-0.35) rectangle (4.4,0.35);
        \node[font=\scriptsize\bfseries, white] at (4.2,0) {R};
        
        % Obstacles (wooden boxes)
        \foreach \x/\y in {0.5/0, 2/-1.2, 2/1.2, -2.3/1.4, -1/0.7, -1/-0.7, -2.3/-1.4} {
            \draw[fill=brown!60, thick] (\x-0.2,\y-0.2) rectangle (\x+0.2,\y+0.2);
        }
        
        % Cubes scattered (small colored squares)
        \foreach \x/\y in {-1.5,0.3, 0.8,0.9, -0.5,-0.8, 1.5,-0.5, -2.0,-0.5} {
            \draw[fill=red!80] (\x,\y) rectangle ++(0.12,0.12);
        }
        \foreach \x/\y in {-0.3,1.5, 1.2,0.2, -1.8,0.8, 0.3,-1.2, -0.8,0.3} {
            \draw[fill=green!80] (\x,\y) rectangle ++(0.12,0.12);
        }
        \foreach \x/\y in {0.5,1.6, -1.2,-1.3, 1.8,0.8, -0.1,0.7, 0.9,-0.9} {
            \draw[fill=blue!80] (\x,\y) rectangle ++(0.12,0.12);
        }
        
        % Arrows showing the task
        \draw[->, ultra thick, green!60!black, dashed] (-2.0,0.8) .. controls (0,1.5) .. (3.4,1.75);
        \draw[->, ultra thick, blue!60!black, dashed] (-1.2,-1.3) .. controls (1,-1.8) .. (3.4,-1.75);
        \draw[->, ultra thick, red!60!black, dashed] (1.5,-0.5) .. controls (2.5,0) .. (3.9,0);
        
        % Legend
        \node[draw, fill=white, rounded corners, font=\scriptsize, text width=2.5cm] at (-3.3,-1.8) {
            \textcolor{brown!60}{$\blacksquare$} Obstáculos\\
            \textcolor{red!80}{$\blacksquare$}\textcolor{green!80}{$\blacksquare$}\textcolor{blue!80}{$\blacksquare$} 15 Cubos
        };
    \end{tikzpicture}
    
    \vspace{0.5em}
    {\Large \textbf{15 cubos} $\rightarrow$ \textbf{3 caixas} por cor}
\end{frame}

\begin{frame}{Restrições}
    \centering
    \begin{tikzpicture}[scale=0.85]
        % Center: YouBot
        \node[draw, circle, fill=blue!30, minimum size=2.5cm, thick, font=\bfseries] (robot) at (0,0) {YouBot};

        % Required (green boxes)
        \node[draw, rectangle, fill=green!40, rounded corners, text width=2cm, align=center, thick, font=\small\bfseries]
            (nn) at (-4.5,1.5) {CNN};
        \node[draw, rectangle, fill=green!40, rounded corners, text width=2cm, align=center, thick, font=\small\bfseries]
            (fuzzy) at (4.5,1.5) {Fuzzy};
        \node[draw, rectangle, fill=green!40, rounded corners, text width=2cm, align=center, thick, font=\small\bfseries]
            (sensors) at (0,3) {LIDAR};

        % Prohibited (red boxes, crossed out)
        \node[draw, rectangle, fill=red!40, rounded corners, text width=1.8cm, align=center, thick, font=\small]
            (gps) at (-4.5,-1.5) {\sout{GPS}};
        \node[draw, rectangle, fill=red!40, rounded corners, text width=1.8cm, align=center, thick, font=\small]
            (tele) at (0,-3) {\sout{Teleop}};
        \node[draw, rectangle, fill=red!40, rounded corners, text width=1.8cm, align=center, thick, font=\small]
            (priv) at (4.5,-1.5) {\sout{Posições}};

        % Arrows
        \draw[->, thick, green!70!black, line width=2pt] (nn) -- (robot);
        \draw[->, thick, green!70!black, line width=2pt] (fuzzy) -- (robot);
        \draw[->, thick, green!70!black, line width=2pt] (sensors) -- (robot);
        \draw[->, thick, red!70, dashed, line width=2pt] (gps) -- (robot);
        \draw[->, thick, red!70, dashed, line width=2pt] (tele) -- (robot);
        \draw[->, thick, red!70, dashed, line width=2pt] (priv) -- (robot);

        % Labels
        \node[font=\scriptsize, green!70!black] at (-2.5,1.8) {obrigatório};
        \node[font=\scriptsize, red!70] at (-2.5,-2) {proibido};
    \end{tikzpicture}
\end{frame}

%=================================================
\section{Sistema de Percepção}
%=================================================

\begin{frame}{Sensores do YouBot}
    \centering
    \begin{columns}[c]
        \begin{column}{0.55\textwidth}
            \centering
            \begin{tikzpicture}[scale=0.75]
                % Robot body (top view) - more detailed
                \draw[fill=gray!20, thick, rounded corners=5pt] (-1.8,-2.5) rectangle (1.8,2.5);
                \node[font=\small\bfseries] at (0,0) {YouBot};
                \node[font=\tiny] at (0,-0.5) {58×38cm};

                % Wheels (mecanum)
                \draw[fill=gray!50, thick] (-2,-2) rectangle (-1.8,-1);
                \draw[fill=gray!50, thick] (-2,1) rectangle (-1.8,2);
                \draw[fill=gray!50, thick] (1.8,-2) rectangle (2,-1);
                \draw[fill=gray!50, thick] (1.8,1) rectangle (2,2);

                % LIDAR rays (180 degrees from front)
                \foreach \a in {-85,-70,-55,-40,-25,-10,5,20,35,50,65,80} {
                    \draw[red!50, ->, line width=0.8pt] (0,2.5) -- ++(\a+90:2.8);
                }
                \node[red!80, font=\scriptsize\bfseries] at (0,5.8) {LIDAR 180°};
                \node[red!60, font=\tiny] at (0,5.4) {180 raios | 5m};

                % Camera FOV cone
                \draw[blue!50, fill=blue!10, line width=1pt] (0,2.5) -- ++(60:2) arc (60:120:2) -- cycle;
                \node[blue!70, font=\scriptsize\bfseries] at (0,4.2) {Câmera};

                % Front distance sensors (3)
                \fill[green!70!black] (-0.9,2.5) circle (0.1);
                \fill[green!70!black] (0,2.5) circle (0.1);
                \fill[green!70!black] (0.9,2.5) circle (0.1);

                % Rear distance sensors (3)
                \fill[green!70!black] (-0.9,-2.5) circle (0.1);
                \fill[green!70!black] (0,-2.5) circle (0.1);
                \fill[green!70!black] (0.9,-2.5) circle (0.1);

                % Lateral sensors (NEW)
                \fill[purple!70] (-1.8,0) circle (0.1);
                \fill[purple!70] (1.8,0) circle (0.1);
                \draw[purple!50, ->, line width=0.8pt] (-1.8,0) -- (-3,0);
                \draw[purple!50, ->, line width=0.8pt] (1.8,0) -- (3,0);

                % Labels
                \node[green!60!black, font=\tiny] at (0,3) {3 front};
                \node[green!60!black, font=\tiny] at (0,-3) {3 rear};
                \node[purple!60, font=\tiny, rotate=90] at (-2.5,0) {lateral};
                \node[purple!60, font=\tiny, rotate=-90] at (2.5,0) {lateral};
            \end{tikzpicture}
        \end{column}
        \begin{column}{0.42\textwidth}
            \begin{tikzpicture}[scale=0.9]
                % LIDAR specs
                \node[draw, rounded corners, fill=red!10, font=\scriptsize, text width=4.5cm, align=left] at (0,3) {
                    \textbf{\textcolor{red!70}{LIDAR}}\\[0.2em]
                    $\bullet$ 180 raios, FOV 180°\\
                    $\bullet$ Range: 0.1--5.0m\\
                    $\bullet$ Atualiza: 32Hz\\
                    $\bullet$ Constrói grade ocupação
                };

                % Camera specs
                \node[draw, rounded corners, fill=blue!10, font=\scriptsize, text width=4.5cm, align=left] at (0,0.8) {
                    \textbf{\textcolor{blue!70}{Câmera RGB}}\\[0.2em]
                    $\bullet$ 320×240 pixels\\
                    $\bullet$ Recognition API\\
                    $\bullet$ Entrada para CNN
                };

                % Distance sensors specs
                \node[draw, rounded corners, fill=green!10, font=\scriptsize, text width=4.5cm, align=left] at (0,-1.3) {
                    \textbf{\textcolor{green!60!black}{Dist. Sensors}}\\[0.2em]
                    $\bullet$ 6 frontais/traseiros\\
                    $\bullet$ 2 laterais (novos)\\
                    $\bullet$ Range: 5cm--2m
                };
            \end{tikzpicture}
        \end{column}
    \end{columns}
\end{frame}

\begin{frame}{Recognition API do Webots}
    \centering
    \begin{tikzpicture}[scale=0.75]
        % Camera sensor
        \node[draw, rectangle, fill=blue!30, rounded corners, minimum width=2cm, minimum height=1.5cm, font=\small\bfseries] (cam) at (0,0) {Câmera};

        % Arrow
        \draw[->, ultra thick] (1.3,0) -- (2.5,0);

        % Recognition module
        \node[draw, rectangle, fill=orange!30, rounded corners, minimum width=3cm, minimum height=2cm, font=\small\bfseries, align=center] (recog) at (4.5,0) {Recognition\\API};

        % Arrow
        \draw[->, ultra thick] (6.3,0) -- (7.5,0);

        % Output structure
        \node[draw, rectangle, fill=green!20, rounded corners, minimum width=4cm, minimum height=3cm, font=\scriptsize, align=left, text width=3.5cm] (output) at (10,0) {
            \textbf{WbCameraRecognitionObject:}\\[0.3em]
            $\bullet$ \texttt{position[3]} (x,y,z)\\
            $\bullet$ \texttt{size[2]} (w,h)\\
            $\bullet$ \texttt{colors[]} (RGB)\\
            $\bullet$ \texttt{model} (nome)
        };

        % Explanation boxes below
        \node[draw, rounded corners, fill=yellow!15, font=\scriptsize, text width=4cm, align=center] at (0,-2.5) {
            \textbf{Entrada:}\\
            Imagem RGB da cena\\
            (320×240 pixels)
        };

        \node[draw, rounded corners, fill=yellow!15, font=\scriptsize, text width=4cm, align=center] at (4.5,-2.5) {
            \textbf{Processo:}\\
            Segmentação por\\
            \texttt{recognitionColors}\\
            definidos nos objetos
        };

        \node[draw, rounded corners, fill=yellow!15, font=\scriptsize, text width=4cm, align=center] at (10,-2.5) {
            \textbf{Saída:}\\
            Posição relativa,\\
            cor e bounding box\\
            de cada objeto
        };

        % Note
        \node[draw, rounded corners, fill=red!10, font=\scriptsize, text width=10cm, align=center] at (5,-4.5) {
            \textbf{Nota:} A Recognition API do Webots usa informação do simulador (ground truth),\\
            por isso usamos CNN própria para classificação de cor -- requisito do projeto
        };
    \end{tikzpicture}
\end{frame}

\begin{frame}{CNN -- MobileNetV3}
    \centering
    \begin{tikzpicture}[scale=0.8]
        % Pipeline visual
        % Input image
        \node[draw, rectangle, fill=blue!20, minimum width=1.5cm, minimum height=1.5cm] (input) at (0,0) {};
        \draw[step=0.3, gray!50, thin] (-0.75,-0.75) grid (0.75,0.75);
        \node[below, font=\scriptsize] at (0,-1) {64x64 RGB};

        % Arrow
        \draw[->, ultra thick] (1.2,0) -- (2.3,0);

        % CNN block
        \node[draw, rectangle, fill=orange!30, rounded corners, minimum width=2.5cm, minimum height=2cm, font=\small\bfseries, align=center] (cnn) at (4,0) {MobileNetV3\\Small};
        \node[below, font=\scriptsize] at (4,-1.3) {pre-treinado};

        % Arrow
        \draw[->, ultra thick] (5.7,0) -- (6.8,0);

        % Classifier
        \node[draw, rectangle, fill=green!30, rounded corners, minimum width=1.8cm, minimum height=1.5cm, font=\small\bfseries] (class) at (8.2,0) {256-3};
        \node[below, font=\scriptsize] at (8.2,-1) {Classificador};

        % Arrow
        \draw[->, ultra thick] (9.5,0) -- (10.6,0);

        % Output - colored bars instead of nested tikz
        \fill[red!70] (11.5,0.4) rectangle (12.5,0.7);
        \fill[green!70] (11.5,-0.15) rectangle (12.5,0.15);
        \fill[blue!70] (11.5,-0.7) rectangle (12.5,-0.4);
        \draw[thick, rounded corners] (11.3,-0.9) rectangle (12.7,0.9);
        \node[below, font=\scriptsize] at (12,-1.3) {Cor};

        % Accuracy badge
        \node[draw, circle, fill=green!50, minimum size=1.5cm, font=\bfseries] at (8.2,2.5) {99.4\%};
        \node[font=\scriptsize] at (8.2,1.5) {Acuracia};
    \end{tikzpicture}
\end{frame}

%=================================================
\section{Navegação Fuzzy}
%=================================================

\begin{frame}{Lógica Fuzzy}
    \centering
    \begin{columns}[c]
        \begin{column}{0.48\textwidth}
            \centering
            % Membership functions (large, clear)
            \begin{tikzpicture}[scale=1.1]
                % Axes
                \draw[->, thick] (0,0) -- (5,0) node[right, font=\small] {dist(m)};
                \draw[->, thick] (0,0) -- (0,2.5) node[above, font=\small] {$\mu$};

                % muito_perto (red)
                \draw[line width=3pt, red!70] (0,2) -- (0.5,2) -- (1.5,0);
                \node[red!70, font=\scriptsize\bfseries] at (0.7,2.3) {muito\_perto};

                % perto (orange)
                \draw[line width=3pt, orange!80] (0.5,0) -- (1.5,2) -- (2.5,0);
                \node[orange!80, font=\scriptsize\bfseries] at (1.5,2.3) {perto};

                % longe (green)
                \draw[line width=3pt, green!70!black] (2,0) -- (3,2) -- (4.8,2);
                \node[green!70!black, font=\scriptsize\bfseries] at (4,2.3) {longe};

                % Ticks
                \node[below, font=\tiny] at (0.8,-0.1) {0.25};
                \node[below, font=\tiny] at (1.7,-0.1) {0.45};
                \node[below, font=\tiny] at (3.2,-0.1) {1.0};
            \end{tikzpicture}
        \end{column}
        \begin{column}{0.48\textwidth}
            \centering
            % Why Fuzzy - big icons
            \begin{tikzpicture}[scale=0.9]
                % Smooth transitions
                \node[draw, circle, fill=blue!30, minimum size=1.8cm, font=\scriptsize\bfseries, align=center] at (0,2) {Suave};
                
                % Handles uncertainty
                \node[draw, circle, fill=orange!30, minimum size=1.8cm, font=\scriptsize\bfseries, align=center] at (2.5,2) {Incerteza};
                
                % Intuitive rules
                \node[draw, circle, fill=green!30, minimum size=1.8cm, font=\scriptsize\bfseries, align=center] at (1.25,0) {Intuitivo};
                
                % Connecting lines
                \draw[thick, gray!50] (0,1.1) -- (1.25,0.9);
                \draw[thick, gray!50] (2.5,1.1) -- (1.25,0.9);
                \draw[thick, gray!50] (0.9,2) -- (1.6,2);
            \end{tikzpicture}
        \end{column}
    \end{columns}
\end{frame}

\begin{frame}{Regras Fuzzy}
    \centering
    \begin{tikzpicture}[scale=0.85]
        % 4 scenarios side by side
        
        % Scenario 1: Emergency - reverse + strafe
        \begin{scope}[shift={(0,0)}]
            \draw[thick, gray!50, rounded corners] (-1.2,-1.2) rectangle (1.2,1.8);
            \draw[fill=blue!30, thick] (-0.3,-0.3) rectangle (0.3,0.5);
            \node[font=\tiny] at (0,0.1) {R};
            \draw[fill=red!60, thick] (-0.2,1) rectangle (0.2,1.4);
            \draw[->, ultra thick, red!70] (0,-0.3) -- (0,-0.9);
            \draw[->, thick, green!60!black] (0.3,0.1) -- (0.9,0.1);
            \node[below, font=\scriptsize\bfseries] at (0,-1.4) {Ré + Strafe};
            \node[above, font=\tiny, red!70] at (0,1.6) {muito perto};
        \end{scope}

        % Scenario 2: Side obstacle - strafe away
        \begin{scope}[shift={(4,0)}]
            \draw[thick, gray!50, rounded corners] (-1.2,-1.2) rectangle (1.2,1.8);
            \draw[fill=blue!30, thick] (-0.3,-0.3) rectangle (0.3,0.5);
            \node[font=\tiny] at (0,0.1) {R};
            \draw[fill=orange!60, thick] (-1,0) rectangle (-0.6,0.6);
            \draw[->, thick, green!60!black] (0.3,0.1) -- (0.9,0.1);
            \node[below, font=\scriptsize\bfseries] at (0,-1.4) {Strafe};
            \node[above, font=\tiny, orange!70] at (0,1.6) {lateral};
        \end{scope}

        % Scenario 3: Aligned - forward fast
        \begin{scope}[shift={(8,0)}]
            \draw[thick, gray!50, rounded corners] (-1.2,-1.2) rectangle (1.2,1.8);
            \draw[fill=blue!30, thick] (-0.3,-0.3) rectangle (0.3,0.5);
            \node[font=\tiny] at (0,0.1) {R};
            \draw[fill=green!60, thick] (-0.15,1.2) rectangle (0.15,1.5);
            \draw[->, ultra thick, green!60!black] (0,0.5) -- (0,1.1);
            \node[below, font=\scriptsize\bfseries] at (0,-1.4) {Avançar};
            \node[above, font=\tiny, green!70!black] at (0,1.6) {alinhado};
        \end{scope}

        % Scenario 4: Big angle - rotate only
        \begin{scope}[shift={(12,0)}]
            \draw[thick, gray!50, rounded corners] (-1.2,-1.2) rectangle (1.2,1.8);
            \draw[fill=blue!30, thick] (-0.3,-0.3) rectangle (0.3,0.5);
            \node[font=\tiny] at (0,0.1) {R};
            \draw[fill=green!60, thick] (0.8,1) rectangle (1.1,1.3);
            \draw[->, thick, orange!70] (0,0.5) arc (90:45:0.7);
            \node[below, font=\scriptsize\bfseries] at (0,-1.4) {Rotacionar};
            \node[above, font=\tiny, orange!70] at (0,1.6) {ângulo grande};
        \end{scope}
    \end{tikzpicture}
\end{frame}

%=================================================
\section{Planejamento A*}
%=================================================

\begin{frame}{Algoritmo A*}
    \centering
    \begin{columns}[c]
        \begin{column}{0.45\textwidth}
            \centering
            % Formula big and clear
            {\Huge $f(n) = g(n) + h(n)$}
            
            \vspace{1em}
            
            \begin{tikzpicture}
                \node[draw, rectangle, fill=blue!20, rounded corners, minimum width=3cm, minimum height=0.8cm] at (0,1) {$g$ = custo até aqui};
                \node[draw, rectangle, fill=orange!20, rounded corners, minimum width=3cm, minimum height=0.8cm] at (0,0) {$h$ = heurística};
                \node[draw, rectangle, fill=green!20, rounded corners, minimum width=3cm, minimum height=0.8cm] at (0,-1) {$f$ = total};
            \end{tikzpicture}
        \end{column}
        \begin{column}{0.5\textwidth}
            \centering
            % A* grid visualization (larger)
            \begin{tikzpicture}[scale=0.7]
                % Grid
                \foreach \x in {0,1,...,8} {
                    \draw[gray!30] (\x,0) -- (\x,6);
                }
                \foreach \y in {0,1,...,6} {
                    \draw[gray!30] (0,\y) -- (8,\y);
                }

                % Obstacles
                \fill[brown!50] (3,2) rectangle (5,4);

                % Start
                \fill[green!70] (1,1) rectangle (2,2);
                \node[font=\bfseries] at (1.5,1.5) {S};

                % Goal
                \fill[red!70] (7,5) rectangle (8,6);
                \node[font=\bfseries, white] at (7.5,5.5) {G};

                % Path
                \draw[blue!70, line width=3pt, ->] (1.5,1.5) -- (1.5,2.5) -- (2.5,3.5) -- (2.5,4.5) -- (3.5,5.5) -- (5.5,5.5) -- (7.5,5.5);

                % Waypoints
                \foreach \x/\y in {1.5/2.5, 2.5/3.5, 2.5/4.5, 3.5/5.5, 5.5/5.5} {
                    \fill[blue!70] (\x,\y) circle (0.15);
                }
            \end{tikzpicture}
        \end{column}
    \end{columns}
\end{frame}

\begin{frame}{Inflação de Obstáculos}
    \centering
    \begin{tikzpicture}[scale=1.2]
        % Original obstacle
        \fill[brown!70] (0,0) rectangle (1.5,1.5);
        \node[font=\small\bfseries, white] at (0.75,0.75) {30cm};
        
        % Inflated area
        \draw[red!50, dashed, line width=2pt] (-0.6,-0.6) rectangle (2.1,2.1);
        \fill[red!20, opacity=0.5] (-0.6,-0.6) rectangle (2.1,2.1);
        \fill[brown!70] (0,0) rectangle (1.5,1.5);
        
        % Robot showing why
        \draw[blue!60, fill=blue!30, thick] (3.5,0.3) rectangle (4.5,1.2);
        \node[font=\tiny\bfseries] at (4,0.75) {YouBot};
        \node[font=\scriptsize] at (4,-0.2) {58×38cm};
        
        % Arrow showing safe path
        \draw[->, ultra thick, green!60!black] (4,-1) -- (4,2.3);
        \node[font=\scriptsize, green!60!black] at (4,2.6) {caminho seguro};
        
        % Dimension arrow
        \draw[<->, thick] (-0.6,-1) -- (2.1,-1);
        \node[font=\scriptsize, below] at (0.75,-1.1) {+30cm margem};
        
        % Label
        \node[draw, rounded corners, fill=yellow!20, font=\scriptsize, text width=3cm, align=center] at (-2,0.75) {
            Inflação = \\[0.2em]
            \textbf{meia-diagonal}\\do robô
        };
    \end{tikzpicture}
\end{frame}

%=================================================
\section{Arquitetura do Sistema}
%=================================================

\begin{frame}{Arquitetura Modular}
    \centering
    \begin{tikzpicture}[scale=0.75]
        % Module boxes
        \node[draw, rectangle, rounded corners, fill=blue!20, minimum width=2.8cm, minimum height=1cm, font=\scriptsize\bfseries] 
            (constants) at (0,3) {constants.py};
        \node[draw, rectangle, rounded corners, fill=green!20, minimum width=2.8cm, minimum height=1cm, font=\scriptsize\bfseries] 
            (routes) at (4,3) {routes.py};
        \node[draw, rectangle, rounded corners, fill=orange!20, minimum width=2.8cm, minimum height=1cm, font=\scriptsize\bfseries] 
            (grid) at (8,3) {occupancy\_grid.py};
        \node[draw, rectangle, rounded corners, fill=purple!20, minimum width=2.8cm, minimum height=1cm, font=\scriptsize\bfseries] 
            (fuzzy) at (12,3) {fuzzy\_navigator.py};
        
        % Main controller
        \node[draw, rectangle, rounded corners, fill=red!30, minimum width=8cm, minimum height=1.5cm, font=\bfseries] 
            (controller) at (6,0) {youbot\_controller.py};
        
        % Entry point
        \node[draw, rectangle, rounded corners, fill=gray!20, minimum width=2.5cm, minimum height=0.8cm, font=\scriptsize\bfseries] 
            (entry) at (6,-2.5) {youbot.py};
        
        % Hardware modules
        \node[draw, rectangle, rounded corners, fill=cyan!20, minimum width=1.8cm, minimum height=0.8cm, font=\tiny\bfseries] 
            (base) at (0,0) {base.py};
        \node[draw, rectangle, rounded corners, fill=cyan!20, minimum width=1.8cm, minimum height=0.8cm, font=\tiny\bfseries] 
            (arm) at (12,0) {arm.py};
        \node[draw, rectangle, rounded corners, fill=yellow!30, minimum width=2.2cm, minimum height=0.8cm, font=\tiny\bfseries] 
            (cnn) at (6,1.8) {color\_classifier.py};
        
        % Arrows
        \draw[->, thick] (constants) -- (controller);
        \draw[->, thick] (routes) -- (controller);
        \draw[->, thick] (grid) -- (controller);
        \draw[->, thick] (fuzzy) -- (controller);
        \draw[->, thick] (cnn) -- (controller);
        \draw[->, thick] (base) -- (controller);
        \draw[->, thick] (arm) -- (controller);
        \draw[->, thick] (entry) -- (controller);
        
        % Labels
        \node[font=\tiny, blue!70] at (0,2.3) {Arena config};
        \node[font=\tiny, green!70!black] at (4,2.3) {Waypoints};
        \node[font=\tiny, orange!70] at (8,2.3) {A* + Grid};
        \node[font=\tiny, purple!70] at (12,2.3) {Fuzzy rules};
        \node[font=\tiny, yellow!70!black] at (6,1.2) {MobileNetV3};
    \end{tikzpicture}
    
    \vspace{0.5em}
    {\small \textbf{Separação de responsabilidades} $\rightarrow$ Código manutenível}
\end{frame}

\begin{frame}{Máquina de Estados}
    \centering
    \begin{tikzpicture}[
        scale=0.95,
        state/.style={draw, rectangle, rounded corners, minimum width=2cm, minimum height=0.9cm, thick, font=\small\bfseries},
        arrow/.style={->, thick, >=stealth}
    ]
        % States (circular layout with RETURN)
        \node[state, fill=yellow!40] (search) at (0,2) {SEARCH};
        \node[state, fill=orange!40] (approach) at (3.5,2) {APPROACH};
        \node[state, fill=green!40] (grasp) at (6.5,1) {GRASP};
        \node[state, fill=purple!40] (tobox) at (6.5,-1) {TO\_BOX};
        \node[state, fill=red!40] (drop) at (3.5,-2) {DROP};
        \node[state, fill=cyan!40] (return) at (0,-2) {RETURN};

        % Transitions
        \draw[arrow, blue!70, line width=1.5pt] (search) -- node[above, font=\tiny] {detectado} (approach);
        \draw[arrow, blue!70, line width=1.5pt] (approach) -- node[right, font=\tiny, pos=0.4] {perto} (grasp);
        \draw[arrow, blue!70, line width=1.5pt] (grasp) -- node[right, font=\tiny] {ok} (tobox);
        \draw[arrow, blue!70, line width=1.5pt] (tobox) -- node[below, font=\tiny] {chegou} (drop);
        \draw[arrow, blue!70, line width=1.5pt] (drop) -- node[below, font=\tiny] {soltou} (return);
        \draw[arrow, blue!70, line width=1.5pt] (return) -- node[left, font=\tiny] {spawn} (search);

        % Self-loop (lost)
        \draw[arrow, red!60, dashed] (approach.north) .. controls (1.75,3.5) and (-1.5,3.5) .. node[above, font=\tiny] {perdido} (search.north);
        
        % Return phases annotation
        \node[draw, rounded corners, fill=cyan!10, font=\tiny, text width=2.5cm, align=left] at (10,0) {
            \textbf{RETURN:}\\
            1. Recuo 60cm\\
            2. Turn-in-place\\
            3. Navigate
        };
    \end{tikzpicture}
\end{frame}

\begin{frame}{Pipeline Completo}
    \centering
    \begin{tikzpicture}[
        scale=0.8,
        box/.style={draw, rectangle, rounded corners, minimum width=2cm, minimum height=0.9cm, thick},
        arrow/.style={->, thick}
    ]
        % Sensors (left)
        \node[box, fill=blue!30, font=\scriptsize\bfseries] (lidar) at (0,2) {LIDAR};
        \node[box, fill=blue!30, font=\scriptsize\bfseries] (camera) at (0,0) {Câmera};
        \node[box, fill=blue!30, font=\scriptsize\bfseries] (dist) at (0,-2) {Dist. Sens.};
        \node[above, font=\small\bfseries, blue!70] at (0,3) {SENSORES};

        % Processing (middle)
        \node[box, fill=orange!30, font=\scriptsize\bfseries] (grid) at (5,2) {Grade};
        \node[box, fill=orange!30, font=\scriptsize\bfseries] (cnn) at (5,0) {CNN};
        \node[box, fill=orange!30, font=\scriptsize\bfseries] (fuzzy) at (5,-2) {Fuzzy};
        \node[above, font=\small\bfseries, orange!70] at (5,3) {PROCESSO};

        % Planning (right-middle)
        \node[box, fill=green!30, font=\scriptsize\bfseries] (astar) at (10,1) {A*};
        \node[box, fill=green!30, font=\scriptsize\bfseries] (fsm) at (10,-1) {FSM};
        \node[above, font=\small\bfseries, green!70!black] at (10,3) {DECISÃO};

        % Output (far right)
        \node[box, fill=red!30, font=\scriptsize\bfseries] (motor) at (14,0) {Motores};
        \node[above, font=\small\bfseries, red!70] at (14,3) {AÇÃO};

        % Arrows
        \draw[arrow] (lidar) -- (grid);
        \draw[arrow] (camera) -- (cnn);
        \draw[arrow] (dist) -- (fuzzy);
        \draw[arrow] (grid) -- (astar);
        \draw[arrow] (cnn) -- (fsm);
        \draw[arrow] (fuzzy) -- (fsm);
        \draw[arrow] (astar) -- (fsm);
        \draw[arrow] (fsm) -- (motor);
    \end{tikzpicture}
\end{frame}

%=================================================
\section{Resultados e Demo}
%=================================================

\begin{frame}{Sequência de Coleta}
    \centering
    \begin{tikzpicture}[scale=0.55]
        % Step 1: Search
        \begin{scope}[shift={(0,0)}]
            \draw[thick, gray!40, rounded corners] (-1.8,-1.3) rectangle (1.8,1.8);
            \draw[fill=blue!40, thick] (-0.4,-0.4) rectangle (0.4,0.4);
            % Scanning rays
            \draw[red!50, thick] (0,0.4) -- (30:1.4);
            \draw[red!50, thick] (0,0.4) -- (60:1.4);
            \draw[red!50, thick] (0,0.4) -- (90:1.4);
            \draw[red!50, thick] (0,0.4) -- (120:1.4);
            \draw[red!50, thick] (0,0.4) -- (150:1.4);
            \draw[fill=red!70] (1,1.2) rectangle (1.25,1.45);
            \node[below, font=\small\bfseries] at (0,-1.6) {1. SEARCH};
        \end{scope}

        % Arrow
        \draw[->, ultra thick, green!60!black] (2.3,0.3) -- (3.2,0.3);

        % Step 2: Approach
        \begin{scope}[shift={(5.5,0)}]
            \draw[thick, gray!40, rounded corners] (-1.8,-1.3) rectangle (1.8,1.8);
            \draw[fill=blue!40, thick] (-0.4,-0.2) rectangle (0.4,0.6);
            \draw[fill=red!70] (0.8,0.8) rectangle (1.05,1.05);
            \draw[->, ultra thick, green!60!black] (0,0.6) -- (0.7,0.9);
            \node[below, font=\small\bfseries] at (0,-1.6) {2. APPROACH};
        \end{scope}

        % Arrow
        \draw[->, ultra thick, green!60!black] (7.8,0.3) -- (8.7,0.3);

        % Step 3: Grasp
        \begin{scope}[shift={(11,0)}]
            \draw[thick, gray!40, rounded corners] (-1.8,-1.3) rectangle (1.8,1.8);
            \draw[fill=blue!40, thick] (-0.4,0) rectangle (0.4,0.8);
            % Arm
            \draw[fill=gray!50, thick] (-0.1,0.8) rectangle (0.1,1.4);
            % Gripper with cube
            \draw[fill=orange!60] (-0.2,1.4) rectangle (0.2,1.55);
            \draw[fill=red!70] (-0.08,1.2) rectangle (0.08,1.38);
            \node[below, font=\small\bfseries] at (0,-1.6) {3. GRASP};
        \end{scope}
    \end{tikzpicture}

    \vspace{0.5em}

    \begin{tikzpicture}[scale=0.55]
        % Step 4: TO_BOX
        \begin{scope}[shift={(0,0)}]
            \draw[thick, gray!40, rounded corners] (-2,-1.5) rectangle (2,1.8);
            \draw[fill=blue!40, thick] (-1.4,0) rectangle (-0.6,0.8);
            % Cube on robot
            \draw[fill=red!70] (-1.1,0.8) rectangle (-0.9,1);
            % Target box
            \draw[fill=red!40, thick] (1.2,0.2) rectangle (1.8,0.8);
            % A* path
            \draw[blue!60, thick, dashed, ->] (-1,0.4) -- (0,0.4) -- (0.5,0.8) -- (1.1,0.5);
            \node[below, font=\small\bfseries] at (0,-1.8) {4. TO\_BOX};
        \end{scope}

        % Arrow
        \draw[->, ultra thick, green!60!black] (2.5,0.3) -- (3.4,0.3);

        % Step 5: Drop
        \begin{scope}[shift={(5.8,0)}]
            \draw[thick, gray!40, rounded corners] (-1.8,-1.5) rectangle (1.8,1.8);
            \draw[fill=blue!40, thick] (-0.4,0) rectangle (0.4,0.8);
            % Box
            \draw[fill=red!40, thick] (0.6,-0.2) rectangle (1.4,0.6);
            % Cube dropping
            \draw[fill=red!70] (0.85,0.15) rectangle (1.15,0.45);
            \draw[->, thick, orange!70] (0.1,0.8) -- (0.8,0.3);
            \node[below, font=\small\bfseries] at (0,-1.8) {5. DROP};
        \end{scope}

        % Arrow
        \draw[->, ultra thick, green!60!black] (8.1,0.3) -- (9,0.3);

        % Step 6: Repeat
        \begin{scope}[shift={(11.3,0)}]
            \draw[thick, gray!40, rounded corners] (-1.8,-1.5) rectangle (1.8,1.8);
            \draw[fill=blue!40, thick] (-0.4,-0.2) rectangle (0.4,0.6);
            \draw[->, thick, blue!60] (0,0.6) arc (90:450:0.5);
            \node[font=\Large\bfseries] at (0,0.2) {$\times$15};
            \node[below, font=\small\bfseries] at (0,-1.8) {6. REPETIR};
        \end{scope}
    \end{tikzpicture}
\end{frame}

\begin{frame}{Navegação A* em Ação}
    \centering
    \begin{tikzpicture}[scale=0.7]
        % Arena grid
        \draw[step=0.5, gray!20, thin] (-4.5,-2.8) grid (4.5,2.8);
        \draw[thick] (-4.5,-2.8) rectangle (4.5,2.8);

        % Obstacles with inflation
        \foreach \x/\y in {0.5/0, 2/-1.2, 2/1.2, -2.3/1.5, -1/-0.7, -1/0.7, -2.3/-1.5} {
            \fill[brown!30, opacity=0.6] (\x-0.5,\y-0.5) rectangle (\x+0.5,\y+0.5);
            \fill[brown!70] (\x-0.25,\y-0.25) rectangle (\x+0.25,\y+0.25);
        }

        % Deposit boxes
        \draw[fill=green!60, thick] (3.5,1.5) rectangle (4.2,2.2);
        \draw[fill=blue!60, thick] (3.5,-2.2) rectangle (4.2,-1.5);
        \draw[fill=red!60, thick] (3.9,-0.35) rectangle (4.4,0.35);

        % Robot
        \draw[fill=blue!60, thick] (-3.5,-0.25) rectangle (-2.7,0.25);
        \node[font=\tiny\bfseries, white] at (-3.1,0) {R};

        % A* path to red box
        \draw[blue!70, line width=3pt, ->]
            (-2.7,0) -- (-1.6,0) -- (-1.6,1.7) -- (0.8,1.7) -- (0.8,0.6) -- (1.6,0.6) -- (2.8,0) -- (3.9,0);

        % Waypoints
        \foreach \x/\y in {-1.6/0, -1.6/1.7, 0.8/1.7, 0.8/0.6, 1.6/0.6, 2.8/0} {
            \fill[blue!70] (\x,\y) circle (0.12);
        }

        % Legend
        \node[draw, fill=white, rounded corners, font=\scriptsize] at (-3,-2.2) {
            \textcolor{brown!50}{$\square$} Inflação \quad
            \textcolor{blue!70}{$\bullet$} Waypoints
        };
    \end{tikzpicture}
\end{frame}

\begin{frame}{Demonstração}
    \centering
    \vspace{1em}

    \begin{tikzpicture}
        \node[draw, rounded corners, fill=blue!10, text width=14cm, align=center, minimum height=6cm] {
            {\Huge \textbf{DEMO}}

            \vspace{1.5em}
            {\Large Execução do robô coletando cubos}
            
            \vspace{1em}
            {\large Webots Simulator}
        };
    \end{tikzpicture}
\end{frame}

%=================================================
\section{Limitações}
%=================================================

\begin{frame}{Limitações dos Algoritmos}
    \centering
    \begin{tikzpicture}[scale=0.85]
        % A* box
        \node[draw, rectangle, rounded corners, fill=green!20, minimum width=4.5cm, minimum height=3.5cm] at (0,0) {};
        \node[font=\bfseries, green!60!black] at (0,1.4) {A*};
        \node[font=\scriptsize, text width=4cm, align=left] at (0,0) {
            $\bullet$ Caminhos próximos a obstáculos\\[0.2em]
            $\bullet$ Muitos nós redundantes\\[0.2em]
            $\bullet$ Suavização limitada\\[0.2em]
            $\bullet$ Lento em mapas grandes
        };

        % Fuzzy box
        \node[draw, rectangle, rounded corners, fill=orange!20, minimum width=4.5cm, minimum height=3.5cm] at (5.5,0) {};
        \node[font=\bfseries, orange!70] at (5.5,1.4) {Fuzzy};
        \node[font=\scriptsize, text width=4cm, align=left] at (5.5,0) {
            $\bullet$ Oscilações em navegação\\[0.2em]
            $\bullet$ Mínimos locais\\[0.2em]
            $\bullet$ Ponto cego $<$40cm\\[0.2em]
            $\bullet$ Fraco em amb. dinâmicos
        };

        % CNN box
        \node[draw, rectangle, rounded corners, fill=blue!20, minimum width=4.5cm, minimum height=3.5cm] at (11,0) {};
        \node[font=\bfseries, blue!70] at (11,1.4) {CNN};
        \node[font=\scriptsize, text width=4cm, align=left] at (11,0) {
            $\bullet$ Trade-off precisão/latência\\[0.2em]
            $\bullet$ Sensível a iluminação\\[0.2em]
            $\bullet$ Requer pré-processamento\\[0.2em]
            $\bullet$ Fallback HSV necessário
        };

        % Mitigation note
        \node[draw, rounded corners, fill=purple!10, font=\scriptsize, text width=13cm, align=center] at (5.5,-2.8) {
            \textbf{Mitigações:} Inflação de obstáculos (A*) | Sensores laterais + escape (Fuzzy) | HSV como fallback (CNN)
        };
    \end{tikzpicture}
\end{frame}

%=================================================
\section{Conclusão}
%=================================================

\begin{frame}{Conclusão}
    \centering
    \begin{tikzpicture}[scale=0.9]
        % Three pillars
        \node[draw, rectangle, rounded corners, fill=blue!30, minimum width=3.5cm, minimum height=2.5cm, font=\small\bfseries, align=center] (cnn) at (0,0) {CNN\\[0.3em]\small 99.4\% acc};
        
        \node[draw, rectangle, rounded corners, fill=orange!30, minimum width=3.5cm, minimum height=2.5cm, font=\small\bfseries, align=center] (fuzzy) at (5,0) {Fuzzy\\[0.3em]\small Navegação\\reativa};
        
        \node[draw, rectangle, rounded corners, fill=green!30, minimum width=3.5cm, minimum height=2.5cm, font=\small\bfseries, align=center] (astar) at (10,0) {A*\\[0.3em]\small Planejamento\\global};
        
        % Integration arrow
        \draw[->, ultra thick, purple!70] (1.8,0) -- (3.2,0);
        \draw[->, ultra thick, purple!70] (6.8,0) -- (8.2,0);
        
        % Result
        \node[draw, rectangle, rounded corners, fill=purple!20, minimum width=12cm, minimum height=1.5cm, font=\large\bfseries] at (5,-3) {
            Integração bem-sucedida $\rightarrow$ Robô autônomo funcional
        };
        
        \draw[->, ultra thick, purple!70] (5,-1.5) -- (5,-2.2);
    \end{tikzpicture}
\end{frame}

\begin{frame}[plain]
    \centering
    \vspace{3em}
    
    {\Huge \textbf{Obrigado!}}
    
    \vspace{2em}
    
    {\Large Perguntas?}
    
    \vspace{3em}
    
    \begin{tikzpicture}
        \node[draw, rounded corners, fill=gray!10, text width=8cm, align=center, font=\small] {
            \textbf{Referências Principais}\\[0.5em]
            Hart et al. (1968) -- A*\\
            Zadeh (1965) -- Fuzzy Sets\\
            Howard et al. (2019) -- MobileNetV3
        };
    \end{tikzpicture}
\end{frame}

\end{document}
